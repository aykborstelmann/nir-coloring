%%%%%%%%%%%%%%%%%%%%%%%%%%%%%%%%%%%
% Language Settings
%%%%%%%%%%%%%%%%%%%%%%%%%%%%%%%%%%%

% To specify the language of your thesis, uncomment the appropriate setting
% which depends on your LaTeX compiler. If you use a modern compiler such as
% LuaLaTeX or XeLaTeX, it is recommended to use the 'polyglossia' package to
% specify the language. If you use LaTeX or PDFLaTeX, 'babel' is recommended. 

% English

% LuaLaTeX or XeLaTeX 
\usepackage{polyglossia}
\setmainlanguage{english} 

% PDFLaTeX 
%\usepackage[english]{babel}

% German 

% LuaLaTeX or XeLaTeX 
%\usepackage{polyglossia}
%\setmainlanguage{german} 

% PDFLaTeX 
%\usepackage[ngerman]{babel}

\usepackage{algorithm}
\usepackage[noend]{algpseudocode}

%%%%%%%%%%%%%%%%%%%%%%%%%%%%%%%%%%%
% Settings for list environments 
%%%%%%%%%%%%%%%%%%%%%%%%%%%%%%%%%%%

\usepackage{enumitem}

\setlist{nosep,topsep=1ex,labelindent=\parindent,listparindent=\parindent}

\setlist[description]{leftmargin=2cm,style=nextline}

\setlist[enumerate,1]{label=(\roman*)}
\setlist[enumerate,2]{label=(\alph*)}
\setlist[enumerate,3]{label=(\arabic*)} 

\setlist[itemize,1]{label=\textbullet}
\setlist[itemize,2]{label=\rule[.2ex]{0.8ex}{0.8ex}}
\setlist[itemize,3]{label=-}

%%%%%%%%%%%%%%%%%%%%%%%%%%%%%%%%%%%
% Useful Packages and Settings 
%%%%%%%%%%%%%%%%%%%%%%%%%%%%%%%%%%%

\usepackage{csquotes} 
\usepackage{amssymb}
\usepackage{amsmath}
\usepackage{tabularx}
\usepackage{rotating}
\usepackage{makecell}
\renewcommand{\tabularxcolumn}[1]{>{\small}m{#1}}
\newcolumntype{Y}{@{} >{\centering\arraybackslash}X @{}}
\usepackage{pgf}
\usepackage{hyperref}
\newcommand{\algorithmautorefname}{Algorithm}


%%%%%%%%%%%%%%%%%%%%%%%%%%%%%%%%%%%
% Demo Packages
%%%%%%%%%%%%%%%%%%%%%%%%%%%%%%%%%%%

% The packages included and settings set here are solely for the demo document
% and can be removed in your thesis 

\usepackage{metalogo} 
\usepackage{fancyvrb}

\RecustomVerbatimEnvironment{Verbatim}{Verbatim}{xleftmargin=\parindent,numbers=left,firstnumber=last,tabsize=4} 

%%%%%%%%%%%%%%%%%%%%%%%%%%%%%%%%%%%
% Citation Package
%%%%%%%%%%%%%%%%%%%%%%%%%%%%%%%%%%%
\usepackage{varioref}
\usepackage{cleveref}
\usepackage[backend=biber,style=authoryear]{biblatex}
\renewcommand*{\nameyeardelim}{\addcomma\space} 


%%%%%%%%%%%%%%%%%%%%%%%%%%%%%%%%%%%%
% Theoremstyles
%%%%%%%%%%%%%%%%%%%%%%%%%%%%%%%%%%%%
\theoremstyle{break}
\theorembodyfont{\normalfont}
\theoremseparator{.}
\theorempreskip{1em}
\theorempostskip{1em}
\theoremsymbol{\ensuremath{\diamond}}

\newtheorem{thm}{Theorem}[chapter]
\newtheorem{lem}[thm]{Lemma}
\newtheorem{prop}[thm]{Proposition}
\newtheorem{cor}[thm]{Corollary}

\newtheorem{exa}[thm]{Example}
\newtheorem{exas}[thm]{Examples}

\newtheorem{prblm}[thm]{Problem}
\newtheorem{prblms}[thm]{Problems}

\newtheorem{quest}{Question}
\newtheorem{quests}{Questions}

\newtheorem{rmk}[thm]{Remark}
\newtheorem{rmks}[thm]{Remarks}

\newtheorem{defn}[thm]{Definition}

\theoremstyle{nonumberplain}
\theoremheaderfont{\itshape}
\theoremsymbol{\rule{1ex}{1ex}}

\newtheorem{proof}{Proof}

\newcommand{\x}{\mathbf{x}}
\newcommand{\y}{\mathbf{y}}
\newcommand{\ce}{\mathbf{c}}

\newcommand\todo[1]{\textcolor{red}{\textbf{TODO} #1}\PackageWarning{TODO:}{TODO: #1}}
% Hide todos:
% \renewcommand\todo[1]{}