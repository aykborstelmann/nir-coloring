\chapter{Evaluations}

\section{Evaluation Metrics}

As NIR colorization has two main application contexts: providing human users more interpretable images and to improve object recognition results by enriching the input with more information.
To assess the effectiveness of both models, we define qualitative evaluation categories and quantitative evaluation metrics based on their ability to achieve these goals.

\subsection{Quantitative Evaluation Metrics}
Due to the unpaired setting, in the quantitative evaluation classic solutions, such as the difference between the absolute intensity values or SSIM \cite{ssim}, cannot be applied. Fortunately, methods for comparing the general image similarity of two sets, such as FID \cite{ttur} or
comparing classification results using an off-the-shelf network, are well known to quantitatively assess the quality of image translation.

\subsubsection*{FID}
To measure how close the generated images are to the real images concerning human perception, the \textit{Fréchet Inception Distance} (FID) \cite{ttur} is a commonly used metric.
It empirically estimates how the human eye perceives images for a set of images and computes the distance between two such set representations.
First, a pre-trained InceptionV3 model evaluates each image of the image set and the activation of the last layer is considered the "human perception approximation".
Then for all activation vectors of the evaluated images in the images set, a multidimensional Gaussian is fitted over those activation vectors.
This is done for two image sets, and later the two Gaussians are compared using the Fréchet distance \cite{ttur}.
Intuitively, if the generated images are realistic, the statistics of features in a classification network should be similar to the real ones.

\subsection{Qualitative Evaluation Methods}
For qualitative evaluation, we define categories on how we determine the superior generated images. These are again based on the two main application contexts.

The first category describes the \textit{naturality} of an image, where the generated images should resemble how humans perceive the world, without artifacts or periodic patterns that can affect visual perception.

The second category refers to the \textit{content preservation} of the input image. This should help humans as well as object detection systems to find and classify animals accurately.

\textit{Hallucinations}, as artifacts or scene elements that are not in the input image, contradict naturality and content preservation, and therefore are undesirable.

\section{Loss-Guided Diffusion and Strongly Guided-Diffusion}
\section{Near-Infrared as Grayscale Approximation}
\section{Diffusion and GAN}
\subsection{Extended Dataset --- Robustness}
\subsection{General Comparison}